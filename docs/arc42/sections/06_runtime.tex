\section{Runtime View}
\label{sec:runtime}

This section describes the behavior of Nazar's building blocks through three key runtime scenarios.

% ============================================================================
\subsection{Runtime Scenario 1: Metric Collection and Storage}
\label{subsec:runtime-scenario1}

This scenario shows how metrics flow from monitored servers into the system.

\begin{figure}[H]
    \centering
    \includegraphics[width=0.85\textwidth]{diagrams/images/sequence_metric_collection.png}
    \caption{Metric Collection and Storage Flow}
    \label{fig:sequence-collection}
\end{figure}

\subsubsection{Interaction Steps}

\begin{enumerate}
    \item Monitored server pushes metrics to API Server via HTTPS POST
    \item API Server stores metrics in TimescaleDB
    \item API Server publishes metrics to RabbitMQ for async processing
    \item API Server returns 202 Accepted to the server
\end{enumerate}

\begin{table}[H]
\centering
\begin{tabularx}{\textwidth}{|L{3cm}|X|}
\hline
\textbf{Component} & \textbf{Role} \\
\hline
Monitored Server & Collects and pushes system metrics \\
\hline
API Server & Receives, validates, and routes metrics \\
\hline
TimescaleDB & Persists time-series metric data \\
\hline
RabbitMQ & Queues metrics for analysis processing \\
\hline
\end{tabularx}
\caption{Scenario 1: Participating Components}
\label{tab:scenario1-blocks}
\end{table}

% ============================================================================
\subsection{Runtime Scenario 2: Anomaly Detection and Alerting}
\label{subsec:runtime-scenario2}

This scenario shows how anomalies are detected and alerts are delivered.

\begin{figure}[H]
    \centering
    \includegraphics[width=0.85\textwidth]{diagrams/images/sequence_anomaly_alerting.png}
    \caption{Anomaly Detection and Alerting Flow}
    \label{fig:sequence-alerting}
\end{figure}

\subsubsection{Interaction Steps}

\begin{enumerate}
    \item Analysis Worker consumes metrics from RabbitMQ
    \item Worker queries TimescaleDB for historical baseline data
    \item Worker analyzes metrics and detects anomalies
    \item Worker stores alert record in TimescaleDB
    \item Worker sends notifications via Slack and Email
    \item DevOps Engineer receives alert notification
\end{enumerate}

\begin{table}[H]
\centering
\begin{tabularx}{\textwidth}{|L{3cm}|X|}
\hline
\textbf{Component} & \textbf{Role} \\
\hline
RabbitMQ & Delivers metric messages to worker \\
\hline
Analysis Worker & Detects anomalies and triggers alerts \\
\hline
TimescaleDB & Provides baseline data and stores alerts \\
\hline
Slack / Email & Delivers notifications to users \\
\hline
\end{tabularx}
\caption{Scenario 2: Participating Components}
\label{tab:scenario2-blocks}
\end{table}

% ============================================================================
\subsection{Runtime Scenario 3: Real-time Dashboard Updates}
\label{subsec:runtime-scenario3}

This scenario shows how the dashboard receives real-time updates via Server-Sent Events (SSE).

\begin{figure}[H]
    \centering
    \includegraphics[width=0.85\textwidth]{diagrams/images/sequence_sse_realtime.png}
    \caption{Real-time Dashboard Updates via SSE}
    \label{fig:sequence-sse}
\end{figure}

\subsubsection{Interaction Steps}

\begin{enumerate}
    \item DevOps Engineer opens the Web Dashboard
    \item Dashboard establishes SSE connection to API Server
    \item API Server queries TimescaleDB for latest metrics
    \item API Server pushes metric data as SSE events
    \item Dashboard updates visualizations in real-time
    \item Connection remains open for continuous updates
\end{enumerate}

\begin{table}[H]
\centering
\begin{tabularx}{\textwidth}{|L{3cm}|X|}
\hline
\textbf{Component} & \textbf{Role} \\
\hline
Web Dashboard & Displays metrics and maintains SSE connection \\
\hline
API Server & Streams real-time updates via SSE \\
\hline
TimescaleDB & Provides metric data for streaming \\
\hline
\end{tabularx}
\caption{Scenario 3: Participating Components}
\label{tab:scenario3-blocks}
\end{table}
